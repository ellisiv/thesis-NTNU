\section{Distance Functions}\label{sec:distance-function}

Distance functions are important for level set methods in general, but they are particularly important for this application. That is because the distance function is a part of the velocity field that drives the surface to our point set.

\begin{definition}
A \textit{distance function}, is the minimal euclidean distance from all points to a point set $\mathcal{V}$. Thus $d(\mathbf{x})$ is defined as 
\begin{equation}
    d(\mathbf{x};\mathcal{V}) = \min_{\mathbf{x}_{v} \in \mathcal{V}} \|\mathbf{x}-\mathbf{x}_{v} \|_2
    \label{eq:unsigned-distance-function}
\end{equation}
\end{definition}

The norm $\|\mathbf{x}-\mathbf{x}_{v} \|$ is positive for all input $\mathbf{x}$ and $\mathbf{x}_{v}$, and thus the distance function is always positive, or \textit{unsigned}.

\todo{Insert proof of $|\nabla d| = 1$?}

The distance function can be the computed to an arbitrary point set, and the notion of a distance function makes intuitively sense. Solving the minimization problem can be time consuming, but as long as $\mathcal{V}\neq \emptyset$, $d(\mathbf{x})$ is uniquely defined everywhere.

The distance function can also be used to measure the distance to a curve. In these cases, one can construct a distance function which in addition displays which regions are inside and outside the curve, by adding a sign. Such a \textit{signed distance function} can thus implicitly describe a curve $\Gamma$ by its zero iso-contour and simultaneously provide information about how every point in the domain relates to the curve $\Gamma$.

\begin{definition}
In a domain \domain including a closed region $\Omega$ with surface $\Gamma$, the signed distance function $u_d$ is defined as
\begin{equation}u_d(\mathbf{x}) = \begin{cases} \, d(\mathbf{x}) \qquad & \text{if } \mathbf{x}\in \Omega\\ -d(\mathbf{x}) \qquad & \text{if } \mathbf{x}\in \mathcal{D} \setminus \Omega\end{cases}
\end{equation}
\end{definition}

In one dimension, this is easily 
