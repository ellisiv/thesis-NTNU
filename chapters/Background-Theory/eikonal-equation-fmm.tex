\subsection{The Eikonal equation}
The Eikonal equation is closely tied with distance functions, or more accurately, it turns out that distance functions are solutions of the Eikonal equation. 

For an expanding front, $\Gamma$, moving outwards with speed $F>0$, we can compute the arrival time at a point $(x, y)$, denoted $T(x, y)$. The initial front is then $T(x, y)=0$, and it grows with rate $F$ outwards, since the distance traveled $d$ given a velocity $F$ and time $T$ is $d=FT$ and thus $\diff d/\diff t=F$ and $1=F \nabla T$ and we get
\begin{equation}
    \begin{aligned}
        &|\nabla T| = \frac{1}{F} \\
        &T(x, y)=0 \qquad \text{for } (x, y) \in \Gamma, 
    \end{aligned}
    \label{eq:eikonal-equation}
\end{equation}
which is the Eikonal equation \cite{sethian1999level}. We observe that for $F=1$, this is $|\nabla T|=1$ and from Proposition \ref{prop:grad-distance}, we see that the gradient of the distance function must satisfy the Eikonal equation \eqref{eq:eikonal-equation}. This is intuitive because the relation between the distance and time is $s=FT$ and this a velocity $F=1$ would mean that the distance traveled must be equal to the travel time used with speed 1.

It is this relation to the distance function that makes the Eikonal equation interesting in this setting. In stead of looking at the construction of a distance problem as a minimization problem, we solve the Eikonal equation. 

\begin{comment}
\begin{enumerate}
    \item Relation to Hamilton Jacobi and Level set formulation
    \item Distance function - stupid approach :-) vs idea behind fast marching - boundary moving with speed 1.
    \item Algorithm for FMM??
\end{enumerate}
\end{comment}


\clearpage