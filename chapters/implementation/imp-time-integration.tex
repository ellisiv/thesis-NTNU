\section{Time Integration and Spatial Discretization}
The time integration step is the part where we integrate our partial differential equation to evolve forward in time. In order to do so, we need a numerical ODE solver for the time integration and we need to discretize the spatial differential operators.

We begin with the discretization of the spatial operators. First note that we have shown in \secref{sec:levelset-methods} that the curvature $\kappa(u)$ in \eqref{eq:our-model-restated} can be written in terms of the spatial derivatives of $u$ as 
\begin{equation*}
    \kappa(u) = \frac{u_{xx} u_y^2- 2u_{xy}u_xu_y + u_{yy}u_x^2}{(u_x^2+u_y^2)^{3/2}}.
\end{equation*}
Also note that the absolute value of the gradient of u from \eqref{eq:our-model-restated} is in two dimensions written as 
\begin{equation*}
    |\nabla u| = (u_x^2+u_y^2)^{1/2}.
\end{equation*}
Because our distance dependent function $f(\distanceVm)$ is purely a function of the distance function and not dependent of $u$, we can write the PDE for the general model \eqref{eq:our-model-restated} as

\begin{equation}
    u_t = \alpha\, (u_x^2+u_y^2)^{1/2} f(\distanceVm) + (1-\alpha)\, \frac{u_{xx} u_y^2- 2u_{xy}u_xu_y + u_{yy}u_x^2}{(u_x^2+u_y^2)^{1/2}}.
\end{equation}

The curvature term makes the equation parabolic and thus a straight forward approach is to discretize the derivatives using a central difference approximation. These can be found in the Appendix \todo{Legg inn og referer}. 

Now, what is left is solving the resulting ODE with respect to time. In our implementation, we have used a simple forward Euler scheme which means that the time steps bounded by the CFL-condition can be as small as 


\begin{equation}
    \Delta t \leq \frac{1}{2} h^2
\end{equation}
\todo{Dette kan ikke stå sånn. Enten må man gjøre mere analyse, eller gi en vagere advarsel om /stabilitet.}

\clearpage