\chapter{Level Set Methods}
\newcommand{\uxt}{$u(\mathbf{x}, t)$ }

A level set formulation is an implicit representation of a closed curve or curves. 
The curve is represented as a constant value of a function defined on a higher
dimensional domain. An example of this is level curves on a map. Provided 
a continuous function describing the hight of the mountain on the whole
domain, the contours joins points of equal elevation. In a cartographic
setting, there are numerous curves representing different elevations that 
are separated by a constant height. Thus contours are effectively describing
the steepness and height, and thus the shape, of the mountain. In a level set
context, we are not interested in the shape of the underlying higher dimensional
function, but we are very much interested in the contours, which we call 
\textit{iso-curves} or \textit{iso-contours}.

The image of contours on a map is useful when understanding iso-curves in a
level set setting also. Since the level set formulation is a way of describing
closed curves, we need to know under which circumstances iso-contours forms 
closed curves. As we are used to from maps, elevation curves are always closed.
\todo{Fullfør. Kontinuitet?}

Now that we have an intuition about iso-curves, we can discuss the principle
of a level set method. The goal is to represent a closed curve or surface, 
$\Gamma(t)$, that moves under the influence of a velocity field $\vv{v}$.
The level set approach to this problem, as first presented in 1987 by 
S.Osher and J. Sethian \cite{MR965860}, is to define a continuous function
\uxt defined on a domain $\Omega$ containing $\Gamma|_{t=0}$.
The domain $\Omega$ is split in two parts by the curve $\Gamma$, the interior
$\mathfrak{I}$ and the exterior $\mathfrak{O}$ \todo{Figur}. The function \uxt
must be constructed in a way that it satisfies the following properties
\begin{align}
    u(\mathbf{x}, t) < 0 \qquad &\text{for } \mathbf{x} \in  \mathfrak{I}(t) \label{eq:interior}\\
    u(\mathbf{x}, t) = 0 \qquad &\text{for } \mathbf{x} \in  \Gamma(t) \label{eq:zero-iso-curve}\\
    u(\mathbf{x}, t) > 0 \qquad &\text{for } \mathbf{x} \in  \mathfrak{O}(t) \label{eq:exterior}.
\end{align}
Now, the curve $\Gamma$ can be described in terms of \uxt by being its 
zero iso-contour. This means that if we can find the proper evolution of 
\uxt we can implicitly track the motion of the curve $\Gamma$.
What we now want is to find the right motion for \uxt to
track the level curve $\Gamma$ flowing in the velocity field $\vv{v}$. Because
we are only interested in the movement of the zero iso-curve of \uxt,
differentiate \eqref{eq:zero-iso-curve} with respect to time to find the movement
of $\Gamma$ when \uxt changes.
\begin{equation}
    (u(\mathbf{x}, t)_t + \nabla u(\mathbf{x}, t) \, \Gamma_t)_{\mathbf{x}\in \Gamma} = 0
    \label{eq:general-u-flow}
\end{equation}
When $\Gamma$ flows in the velocity field $\vv{v}$, its time derivative has to be 
defined as
\begin{equation}
    \Gamma_t = \vv{v}\cdot \vv{n}_{\Gamma} = v_n \vv{n}_{\Gamma}.
\end{equation}
In addition, we can see from \eqref{eq:interior}-\eqref{eq:exterior} that the gradient 
of $u$ at the curve $\Gamma$ is always pointing in the direction of the normal vector of 
$\Gamma$, $\vv{n}_{\Gamma}$. Thus we can write 
\begin{equation}
    \vv{n}_{\Gamma} = -\frac{\nabla u}{|\nabla u|}
\end{equation}

Inserting everything back into \eqref{eq:general-u-flow}, we get the equation
\begin{equation}
    u_t - v_n\, |\nabla u| = 0
    \label{eq:general-level-set}
\end{equation}

What to discuss around the method?
\begin{enumerate}
    \item increased complexity for solving PDE on entire domain. How to fix
    \item Topological flexibility. Situations that are now easily handled.
    \item Existence of viscosity solution (?????). In that case, I must read.
\end{enumerate}

\newpage
\section{Level Set Formulation for Surfaces Approaching Unstructured and Irregular Point Sets}
Going back to our problem, we have a set of points sampled from a surface and we want to
reconstruct this surface as a smooth curve. Doing this using
a level set method means that given an initial curve $\Gamma_0$, we want to find the 
appropriate velocity field in \eqref{eq:general-level-set} such that the surface moves
closer and closer to the point set. The problem also includes a certain smoothness to 
the surface that needs to be preserved during this motion.


