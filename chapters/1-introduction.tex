\chapter{Introduction} \label{chap:introduction}
This master thesis is based on a modeling problem described by Claise and Frey in a paper from 2011\cite{Claisse-Frey}. The paper proposed a nonlinear PDE model for reconstructing a regular surface from unstructured, sampled data with use of a level set formulation. The paper and its mathematical model is the starting point of this thesis, which is a study of the theoretical background and implementation of the purposed model in addition to a proposal for a model with different behavior and strengths. 

The background for this study is an ongoing project with \todo{hvem står bak prosjektet?} with the goal of estimating bedrock topography and the geometry of the sediment thickness. This is a shape reconstruction problem, where there are taken samples of the sediment thickness, and from these measurements we want the best possible approximation of the total bedrock topography. The object of interest, in this situation the bedrock, can only be observed implicitly through measurements and thus we know little about the object beforehand.

    Because of the practical application, we cannot assume much about our data points, and that is why it is said that they are \textit{irregular} and \textit{unstructured}. Unstructured, because we do not know anything about the numbering of the points, say in two spatial dimensions, we do not know which points to draw the curve between to approximate the shape. Irregular, because there is no assumption on how the samples are placed, if they are lumped together or far from each other.

The purposed model which is studied in this thesis is thus simple and general and considers only that we want a curve or surface that lie close the the data points, meaning minimal distance, and the curve should be smooth in the sense of minimal curvature. Considering that this thesis is only an investigation on how the purposed model would behave, and thus a study on a possible approach for the bedrock topology problem, we restrict ourselves only to the two dimensional situation in order to free more time to analysis in stead of implementation. 

Our mathematical model is a partial differential equation, constructed from a \textit{level set formulation}. Level set methods was introduced by Osher and Sethian\cite{Osher-Sethian} as an implicit approach to track a moving front. By representing the front implicitly topological changes like merging or braking would be handled naturally. Many have used this paper as a stepping stone and applied level set methods to a variety of physical applications like multiphase phase flow \cite{SUSSMAN1994146} \cite{ZHAO1996179}, crystal growth \cite{sethian1999level} and in image applications like image enhancement and noise reduction \cite{sethian1999level} \cite{RUDIN1992259} and shape detection \cite{368173} \cite{sethian1999level} \cite{5754584} among other things.

This thesis will provide an introduction to general level set methods and distance functions in \chapref{chap:background-theory} containing an implicit derivation, and a proof that explicit tracking of a curve, and implicit tracking through a level set method provides the same result.

\chapref{chap:modeling} will explain the background for modeling a level set formulation to a set of sampled points in addition to some analysis of the behavior in a simplified 1D situation and how parameter choises affect the results.

In \chapref{chap:implementation}, we provide the background for the numerical implementation, and present the main implementational challenges and possible solutions. From this, we present results of simulations in \chapref{chap:results} using different parameter choices and affirming the analysis from \chapref{chap:modeling}.

Finally, in \chapref{chap:concluding-remarks} we present interesting aspects to look further into and discuss the applicability of the level set method for general curve reconstruction and our specific bedrock approximation problem.
%\section{A Mathematical Problem Formulation}
\begin{figure}
    \centering
    \includegraphics{figures/tikz-figures/initial-problem.tex}
    \caption[Problem description]{Caption}
    \label{fig:problem-description}
\end{figure}
Consider a set of points,$v \in \pointsetm$, where $v \in \realspacem^2$ as pictured in \figref{fig:problem-description}. The approach is to construct an initial curve, $\Gamma_0$, encircling the point set which is to be incrementally improved according to given quality measures. 

The incremental improvement is done by moving the curve with a speed dependent on the shape and the distance to the nearest point. Thus the approach is solving a partial differential equation with the starting curve as initial condition. The PDE is a Hamilton-Jacobi type equation which is defined on a subspace, \domain, of $\realspacem^2$ including the point set. The PDE governs the motion of a functional $u : \realspacem^2 \to \realspacem$ with the curve as zero level set. The PDE is defined as follows

\begin{tcolorbox}[title=Mathematical formulation of our PDE model]
\begin{alignat}{2}
    &u_t = |\nabla \uxtm | \big[ (1-\alpha) \kappa(u)(\mathbf{x}, t) + \alpha \sigma(\mathbf{x}, t) d(\mathbf{x})  \big] & \text{for }\mathbf{x}\in \domainm \\
    &u(\mathbf{x}, 0) = u_0 & \text{for }\mathbf{x}\in \domainm\\
    &u_0(\Gamma_0) = 0  & \\
    &\nabla_{\vv{n}} u = 0 &\text{for } \mathbf{x} \in \partial\domainm
\end{alignat}
\end{tcolorbox}



\begin{comment}
We are studying a surface reconstruction problem, where we are given a set of sample points, \pointset, which are spatial points belonging to a closed surface, $\Gamma$. The points are \textit{unstructured} which means that no information is given regarding the order of the points. In addition the samples of the surface could contain some measuring errors, or noise that would make the samples \textit{irregular}. The task is then to reconstruct the original surface using the sample points.

The approach in this thesis is to define two quality measures, curvature and distance from the surface to the point set, and 



In this thesis, the quality of the surface is measured only by its curvature and closeness to the
points. The curvature is a measure of the variation of the surface, or more specific, how much the
angle between the tangent and a constant vector changes over the surface. 

The strategy is constructing a PDE based model, which given an initial guess, continuously perturbs
the surface in a direction that increases the quality measure. This means we introduce a partial 
differential equation with a 
\end{comment}
\begin{comment}
** How can this theory be applied in the real world?

** What are the strengths and weaknesses?

** How is this thesis structured?

** Further work



From an unstructured set of points, $V$, we want to find a closed surface that approaches the point set as closely as possible while also requiring some extent of smoothness. Such a method could be useful when the surface can not be directly observed, but only measured pointwise. An example of this could be modelling geological formations underground or ...

From realistic samples the data would be susceptible to measurement errors and thus a model should be able to handle noisy point sets. However, for the resulting surface to make any sense, the points should not be arbitrarily distributed in the domain, but follow some pattern resembling a surface.

There are several approaches to construct methods solving this problem. *Use this section, or maybe more? To sum up and explain weaknesses and strengths of these approaches.

In this thesis, we consider a PDE-driven approach to form a surface approximating the point set. Here, a surface, or a front, moves with a speed that is dependent on the distance to the point set in addition to the smoothness of the curve in that point. In contrast to a surface tracking procedure, where we start with an initial curve, discretise it and track the movement of the surface points, we will use an approach called a \textit{level set method}. This method describes the surface as an iso-curve of a higher dimensional functional defined on the entire domain. 

The level set method would use more computational power compared to a front tracking method because it also considers the environment around the front/surface. Fortunately the computational costs also comes with increased flexibility. The front tracking approach suffers from some shortcomings when it comes to topological changes like collisions or splitting of surfaces. In addition, front tracking techniques would not be able to model  appearing or disappearing surfaces. Using an implicit definition of the surface using iso-curves of a higher dimensional functional, none if these issues would cause problems. 
\end{comment}

%More specific statements about what is studied

%Statements indicate the need for more research/finding the gap

%Giving the purpose/objectives for the main activity/findings
 
%A "map" presenting how the text is presented